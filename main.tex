%%%%%%%%%%%%%%%%%%%%%%%%%%%%%%%%%%%%%%%%%
% Currículo de Desenvolvedor
% Modelo LaTeX
% Versão 1.0 (28/1/19)
%
% Este modelo é originário de:
% http://www.LaTeXTemplates.com
%
% Autores:
% Jan Vorisek (jan@vorisek.me)
% Baseado em um modelo por Jan Küster (info@jankuester.com)
% Modificado para LaTeX Templates por Vel (vel@LaTeXTemplates.com)
%
% Licença:
% A Licença MIT (consulte o arquivo LICENSE incluído)
%
%%%%%%%%%%%%%%%%%%%%%%%%%%%%%%%%%%%%%%%%%

%----------------------------------------------------------------------------------------
%	PACOTES E OUTRAS CONFIGURAÇÕES DO DOCUMENTO
%----------------------------------------------------------------------------------------

\documentclass[9pt]{developercv} % Tamanho de fonte padrão, valores de 8-12pt são recomendados
\definecolor{bleu_cite}{RGB}{0, 102, 204}
%----------------------------------------------------------------------------------------

\begin{document}

%----------------------------------------------------------------------------------------
%	TÍTULO E INFORMAÇÕES DE CONTATO
%----------------------------------------------------------------------------------------

\begin{minipage}[t]{0.55\textwidth} % 45% da largura da página para o nome
	\vspace{-\baselineskip} % Necessário para alinhar verticalmente minipages
	
	% Se o seu nome for muito curto, use apenas uma das linhas abaixo
	% Se o seu nome for muito longo, reduza o tamanho da fonte ou torne a minipage mais larga e reduza as outras proporcionalmente
	\colorbox{black}{{\fontsize{24}{26}\textcolor{white}{\textbf{\MakeUppercase{Ueslei Adriano Sutil}}}}} % Primeiro nome
	
	\vspace{3pt}
	
	{\Huge \textbf{Cientista de Dados }} % Cargo ou título atual
\end{minipage}
\begin{minipage}[t]{0.235\textwidth} 
	\vspace{-\baselineskip} % Necessário para alinhar verticalmente minipages
	
	% O primeiro parâmetro é o nome do ícone FontAwesome, o segundo é o tamanho da caixa e o terceiro é o texto
	% Outros ícones podem ser encontrados consultando fontawesome.pdf (fornecido com o modelo) e usando a palavra após \fa no comando para o ícone desejado
	\icon{MapMarker}{12}{São Paulo, SP}\\
	\icon{Phone}{12}{+55 12 99144-2815}\\
	\icon{Laptop}{12}{\textcolor{bleu_cite}{\href{https://www.uesleisutil.com}{uesleisutil.com}}}\\

\end{minipage}
\begin{minipage}[t]{0.275\textwidth} % 27.5% da largura da página para a segunda linha de ícones
	\vspace{-\baselineskip} % Necessário para alinhar verticalmente minipages
	
	% O primeiro parâmetro é o nome do ícone FontAwesome, o segundo é o tamanho da caixa e o terceiro é o texto
	% Outros ícones podem ser encontrados consultando fontawesome.pdf (fornecido com o modelo) e usando a palavra após \fa no comando para o ícone desejado
	\icon{Linkedin}{12}{\textcolor{bleu_cite}{\href{https://www.linkedin.com/in/usutil/}{linkedin.com/in/usutil}}}\\
	\icon{At}{12}{\textcolor{bleu_cite}{\href{mailto:ueslei@outlook.com}{ueslei@outlook.com}}}\\
	\icon{Code}{12}{\textcolor{bleu_cite}{\href{https://kaggle.com/uesleisutil}{kaggle.com/uesleisutil}}}\\
\end{minipage}

%----------------------------------------------------------------------------------------
%	INTRODUÇÃO, HABILIDADES E TECNOLOGIAS
%----------------------------------------------------------------------------------------
\cvsect{\faUser\hspace{0.3cm} Sobre Mim}

\begin{minipage}[t]{1\textwidth} % 40% da largura da página para o texto de introdução
	Com mais de dez anos de experiência como Cientista de Dados, dediquei minha carreira ao desenvolvimento de modelos numéricos
	e condução de projetos científicos. Atualmente utilizo análises preditivas para melhorar a comunicação com os clientes e fornecer
	insights decisivos para a alta administração sobre o comportamento e o ciclo de vida do cliente e coordeno o refinamento e priorização
	das atividades da equipe que atuo.
\end{minipage}
\hspace{-0.3cm}
\hfill % Espaço em branco entre

\begin{center}
    \bubbles{4/Python, 4/ML \& DL, 4/SQL, 4/Tableau, 4/PyTorch, 4/AWS, 3.4/LaTeX, 3.4/Agile, 3/Salesforce, 3/R, 3/Quicksight}
\end{center}

%----------------------------------------------------------------------------------------
%	EXPERIÊNCIA
%----------------------------------------------------------------------------------------
\vspace{-0.2cm}
\cvsect{\faBriefcase\hspace{0.3cm} Experiência Profissional}

\begin{entrylist}
	\entry
		{2022 -- Atual\\\footnotesize{CLT}}
		{Cientita de Dados Sênior}
		{\href{https://www.esfera.com.vc}{\textcolor{bleu_cite}{\textbf{Esfera Fidelidade - Uma Empresa do Santander}}}}
		{Ciência de dados aplicada em Customer Relationship Management (CRM) e com atuação em:\\
		• Modelos de Machine e Deep Learning (e.g., Clusterização, Next Best Offer e Previsão de Risco);\\
		• Estudos sobre o comportamento do cliente com foco em ativação e retenção;\\
		• Coordenação e planejamento das atividades da equipe de Analytics e Data Science com foco principal no refinamento e priorização das tarefas através de metodologias ágeis;\\
		• Criação de dashboards executivos;\\
		• Ideação de testes de Growth e Testes A/B.\\
        \texttt{Python}\slashsep\texttt{SQL}\slashsep\texttt{Clustering}\slashsep\texttt{NLP}\slashsep\texttt{Classification}\slashsep\texttt{Regression}\slashsep\texttt{AWS Quicksight}}
		\\\entry
		{2021 --  2022\\\footnotesize{CLT}}
		{Cientista de Dados Pleno}
		{\href{https://www.esfera.com.vc}{\textcolor{bleu_cite}{\textbf{Esfera Fidelidade - Uma Empresa do Santander}}}}
		{Ciência de dados aplicada em Customer Relationship Management (CRM) e com atuação em:\\
		• Modelos de Machine Learning com foco na segmentação do cliente (e.g. Previsão de Consumo do Cliente e Next Best Channel);\\
		• Criação e manutenção de painéis e indicadores operacionais;\\
		\texttt{Python}\slashsep\texttt{Jupyter}\slashsep\texttt{AWS Sagemaker}\slashsep\texttt{SQL}\slashsep\texttt{Tableau}\slashsep\texttt{Salesforce}}
		\\\entry
		{2017 -- 2021\\\footnotesize{Bolsa CNPq III}}
		{Pesquisador Assistente}
		{\href{https://www.gov.br/inpe/pt-br}{\textcolor{bleu_cite}{\textbf{Instituto Nacional de Pesquisas Espaciais (INPE)}}}}
		{• Construção da ferramenta modelo2roms. Projeto de código aberto disponível no \href{https://github.com/uesleisutil/model2roms}{\textcolor{bleu_cite}{\textbf{Github}}}; \\	
		• Implementação dos modelos WaveWatch 3 e ROMS Sea-Ice em um supercomputador;\\
		%• Construção do \href{https://loa-inpe.github.io/}{\textcolor{bleu_cite}{\textbf{site}}} para o Laboratório de Estudos do Oceano e da Atmosfera. \\
        \texttt{Python}\slashsep\texttt{LaTeX}\slashsep\texttt{Fortran}\slashsep\texttt{C++}\slashsep\texttt{Statistics}\slashsep\texttt{Git}}
		\\\entry
		{2016 -- 2017\\\footnotesize{Bolsa CNPq II}}
		{Pesquisador Assistente}
		{\href{https://www.gov.br/inpe/pt-br}{\textcolor{bleu_cite}{\textbf{Instituto Nacional de Pesquisas Espaciais (INPE)}}}}
		{• Implementação do sistema de modelagem numérica COAWST em um supercomputador; \\
		 • Modelagem hidrodinâmica e atmosférica para estudar a interação oceano-atmosfera no Atlântico Sul. 
		 \texttt{Python}\slashsep\texttt{Statistics}\slashsep\texttt{Fortran}\slashsep\texttt{C++}\slashsep\texttt{MATLAB}}
\end{entrylist}

%----------------------------------------------------------------------------------------
%	EDUCAÇÃO
%----------------------------------------------------------------------------------------
\vspace{-0.5cm}
\cvsect{\faGraduationCap\hspace{0.3cm} Educação}

\begin{entrylist}
	\entry
		{2023 -- Atual}
		{MBA em Ciência de Dados \& IA}
		{Faculdade de Informática e Administração Paulista (FIAP)}
		{}
		\entry
		{2014 -- 2016}
		{Mestrado em Sensoriamento Remoto}
		{Universidade Federal do Rio Grande do Sul (UFRGS)}
		{\href{https://lume.ufrgs.br/handle/10183/171223}{\textcolor{bleu_cite}{\textbf{Tese}}}: \textit{Interação Oceano-Atmosfera em um Ciclone Extratropical no Atlântico Sudoeste}.}
	\entry
		{2009 -- 2013}
		{Bacharelado em Oceanografia}
		{Universidade Federal do Paraná (UFPR)}
		{\href{http://doi.org/10.13140/RG.2.2.15184.35847}{\textcolor{bleu_cite}{\textbf{Monografia}}}: \textit{Variabilidade da Temperatura da Superfície do Mar durante um evento de precipitação extrema em Santa Catarina}. 
		\\ \href{http://doi.org/10.13140/RG.2.2.25250.68800}{\textcolor{bleu_cite}{\textbf{Iniciação Científica}}}: \textit{Aspectos Meteorológicos do evento extremo de novembro de 2008 em Santa Catarina}.
		}
\end{entrylist}

%----------------------------------------------------------------------------------------
%	INFORMAÇÕES ADICIONAIS
%----------------------------------------------------------------------------------------
\vspace{-0.4cm}

\begin{minipage}[t]{0.3\textwidth}
	\vspace{-\baselineskip} % Necessário para alinhar verticalmente minipages
	
	\cvsect{\faBook\hspace{0.3cm} Publicações Científicas}
	
	\textbf{Livro} %\\
	\textcolor{gray}{}
	 \begin{itemize}
		\item \small\textcolor{bleu_cite}{\href{http://mtc-m21c.sid.inpe.br/col/sid.inpe.br/mtc-m21c/2020/10.02.15.11/doc/publicacao.pdf}{Guia do Usuário COAWST - 3ª Ed.}}
	 \end{itemize}
	 \textbf{Artigo}
	 \begin{itemize}
		\item \small\textcolor{bleu_cite}{\href{https://www.nature.com/articles/s41598-021-89985-9}{Modificações induzidas por redemoinhos oceânicos nos fluxos de calor e CO2 ar-mar no Confluência Brasil-Malvinas}}
	 \end{itemize}  
\end{minipage}
\hfill
\begin{minipage}[t]{0.3\textwidth}
	\vspace{-\baselineskip} % Necessário para alinhar verticalmente minipages

	\cvsect{\faCommentsO\hspace{0.3cm} Idiomas}
	\vspace{-0.3cm}
	\begin{itemize}
	\itemsep0em
	\item \textbf{Português} - Nativo
	\item \textbf{Inglês} - Proficiente
	\item \textbf{Italiano} - Intermediário 
	\end{itemize}
	\vspace{0.2cm}
	
    \centering\includegraphics[width=0.25\textwidth]{qrcode.png}\\
	\centering\tiny Atualizado em 14 de Abril de 2024
\end{minipage}
\hfill
\begin{minipage}[t]{0.3\textwidth}
	\vspace{-\baselineskip} % Necessário para alinhar verticalmente minipages

	\cvsect{\faChild\hspace{0.3cm} Trabalho Voluntário}

	\textbf{\textcolor{bleu_cite}{\href{https://www.casaum.org/}{Casa 1}} } \hspace{1.cm} Jan 2019 --  Dez 2019\\
	\textcolor{gray}{Parcerias e Empregabilidade}\\\\
	\textbf{Engaja} \hspace{0.95cm} Jan 2023 --  Mar 2024\\
	\textcolor{gray}{Grupo de Diversidade e Inclusão}
\end{minipage}


%----------------------------------------------------------------------------------------

\end{document}
