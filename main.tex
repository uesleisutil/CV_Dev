%%%%%%%%%%%%%%%%%%%%%%%%%%%%%%%%%%%%%%%%%
% Developer CV
% LaTeX Template
% Version 1.0 (28/1/19)
%
% This template originates from:
% http://www.LaTeXTemplates.com
%
% Authors:
% Jan Vorisek (jan@vorisek.me)
% Based on a template by Jan Küster (info@jankuester.com)
% Modified for LaTeX Templates by Vel (vel@LaTeXTemplates.com)
%
% License:
% The MIT License (see included LICENSE file)
%
%%%%%%%%%%%%%%%%%%%%%%%%%%%%%%%%%%%%%%%%%

%----------------------------------------------------------------------------------------
%	PACKAGES AND OTHER DOCUMENT CONFIGURATIONS
%----------------------------------------------------------------------------------------

\documentclass[9pt]{developercv} % Default font size, values from 8-12pt are recommended
\definecolor{bleu_cite}{RGB}{0, 102, 204}
%----------------------------------------------------------------------------------------

\begin{document}

%----------------------------------------------------------------------------------------
%	TITLE AND CONTACT INFORMATION
%----------------------------------------------------------------------------------------

\begin{minipage}[t]{0.55\textwidth} % 45% of the page width for name
	\vspace{-\baselineskip} % Required for vertically aligning minipages
	
	% If your name is very short, use just one of the lines below
	% If your name is very long, reduce the font size or make the minipage wider and reduce the others proportionately
	\colorbox{black}{{\fontsize{24}{26}\textcolor{white}{\textbf{\MakeUppercase{Ueslei Adriano Sutil}}}}} % First name
	
	\vspace{6pt}
	
	{\Huge \textbf{Oceanógrafo}} % Career or current job title
\end{minipage}
\begin{minipage}[t]{0.235\textwidth} % 27.5% of the page width for the first row of icons
	\vspace{-\baselineskip} % Required for vertically aligning minipages
	
	% The first parameter is the FontAwesome icon name, the second is the box size and the third is the text
	% Other icons can be found by referring to fontawesome.pdf (supplied with the template) and using the word after \fa in the command for the icon you want
	\icon{MapMarker}{12}{São José dos Campos}\\
	\icon{Phone}{12}{+55 12 98706 9684}\\
	\icon{At}{12}{\textcolor{bleu_cite}{\href{mailto:ueslei@outlook.com}{ueslei@outlook.com}}}\\	
\end{minipage}
\begin{minipage}[t]{0.275\textwidth} % 27.5% of the page width for the second row of icons
	\vspace{-\baselineskip} % Required for vertically aligning minipages
	
	% The first parameter is the FontAwesome icon name, the second is the box size and the third is the text
	% Other icons can be found by referring to fontawesome.pdf (supplied with the template) and using the word after \fa in the command for the icon you want
	\icon{Globe}{12}{\textcolor{bleu_cite}{\href{https://www.uesleisutil.com.br}{uesleisutil.com.br}}}\\
	\icon{Github}{12}{\textcolor{bleu_cite}{\href{https://github.com/uesleisutil}{github.com/uesleisutil}}}\\
	\icon{Twitter}{12}{\textcolor{bleu_cite}{\href{https://twitter.com/uesleisutil}{@uesleisutil}}}\\
\end{minipage}

%----------------------------------------------------------------------------------------
%	INTRODUCTION, SKILLS AND TECHNOLOGIES
%-----------------------------------------------------Actively
\cvsect{Sobre mim}

\begin{minipage}[t]{0.6\textwidth} % 40% of the page width for the introduction text

Sou um oceanógrafo físico com habilidades em linguagem de programação e análise e visualização de dados. Atuo ativamente em desenvolvimento utilizando Python.
\end{minipage}
\hspace{0.3cm}
\hfill % Whitespace between
\begin{minipage}[t]{0.5\textwidth} % 50% of the page for the skills bar chart
	
	\vspace{-\baselineskip} % Required for vertically aligning minipages
	\vspace{-0.9cm}
	\begin{barchart}{5.5}
		\baritem{Python}{100}
		\baritem{Django}{95}	
		\baritem{React}{93}
		\baritem{Flask}{97}	
		\baritem{Git}{98}
	\end{barchart}
\end{minipage}
\vspace{-0.1cm}
\begin{center}
	\bubbles{4/scikit-learn, 4/numpy, 3/UX Design, 3.5/git, 3.5/Bootstrap, 4/LaTeX,3.7/Javascript,3.7/HTML, 3.7/CSS+SASS, 3.7/Node.js,2/Kubernetes}
\end{center}

%----------------------------------------------------------------------------------------
%	EXPERIENCE
%----------------------------------------------------------------------------------------
\vspace{-0.2cm}
\cvsect{Experiência profissional}

\begin{entrylist}
	\entry
		{2017 -- 2021\\\footnotesize{Bolsa CNPq}}
		{Pesquisador Assistente}
		{Instituto Nacional de Pesquisas Espaciais (INPE)}
		{• Publicação de livros e artigos sobre modelagem numérica e Python ; \\
		• Treinamento de pessoal em modelagem munérica e Python através de cursos; \\
		• Construção da \textit{toolbox} model2roms, baseada em Python e Fortran,  para criar arquivos necessários para executar o modelo numérico ROMS.  Projeto \textit{open-source} disponível no \href{https://github.com/uesleisutil/model2roms}{\textcolor{bleu_cite}{\textbf{Github}}}; \\	
		• Implementação dos modelos WaveWatch 3 e ROMS Sea-Ice em supercomputador;\\
		• Construção do site do Laboratório de Estudos do Oceano e da Atmosfera. Acesse o site \href{https://loa-inpe.github.io/}{\textcolor{bleu_cite}{\textbf{aqui}}}. \\
		\texttt{Python}\slashsep\texttt{React}\slashsep\texttt{Flask}\slashsep\texttt{Django}\slashsep\texttt{Big data}\slashsep\texttt{Fortran}\slashsep\texttt{C++}\slashsep\texttt{Git}\slashsep\texttt{Bootstrap}\slashsep\texttt{HTML/CSS}}
	\entry
		{2016 -- 2017\\\footnotesize{Bolsa CNPq}}
		{Pesquisador Assistente}
		{Instituto Nacional de Pesquisas Espaciais (INPE)}
		{•  Implementação do sistema de modelagem numérica COAWST em um supercomputador; \\
		 • Modelagem hidrodinâmica e atmosférica usando para estudar eventos climáticos extremos e interação oceano-atmosfera no Atlântico Sudoeste. 
		 \\ \texttt{Python}\slashsep\texttt{Big data}\slashsep\texttt{Fortran}\slashsep\texttt{C++}\slashsep\texttt{MATLAB}}
\end{entrylist}

%----------------------------------------------------------------------------------------
%	EDUCATION
%----------------------------------------------------------------------------------------
\vspace{-0.5cm}
\cvsect{Educação}

\begin{entrylist}
		\entry
		{2020 -- 2021}
		{Dev Full Stack}
		{HSM University}
		{ \texttt{Javascript}\slashsep\texttt{HTML+CSS/SASS} \slashsep\texttt{Webservice Client e Server} \slashsep\texttt{Angular}\slashsep\texttt{Node.js} \slashsep\texttt{UX design} \slashsep\texttt{Banco de dados} }
		\entry
		{2020 -- 2020}
		{Machine Learning A-Z™: Hands-On Python \& R In Data Science}
		{Udemy}
		{Certificado disponível \href{https://www.udemy.com/certificate/UC-06fff362-ee67-467b-a658-df37cfef5368/}{\textcolor{bleu_cite}{\textbf{aqui}}}. \\ \texttt{Python}\slashsep\texttt{R} \slashsep\texttt{Tensorflow}\slashsep\texttt{numpy}\slashsep\texttt{scikit-learn} }
		\entry
		{2014 -- 2016}
		{Mestrado em Sensoriamento Remoto}
		{Universidade Federal do Rio Grande do Sul}
		{\textbf{Dissertação:} \textit{Estudo da Interação Oceano-Atmosfera em um Ciclone Extratropical no Atlântico Sudoeste: uma abordagem numérica em altíssima resolução}. Disponível para consulta \href{https://lume.ufrgs.br/handle/10183/171223}{\textcolor{bleu_cite}{\textbf{aqui}}}. \\  \texttt{Python}\slashsep\texttt{Big data} \slashsep\texttt{NCL}\slashsep\texttt{Fortran}\slashsep\texttt{C++} \slashsep\texttt{MATLAB} \slashsep\texttt{Linux}    }
	\entry
		{2009 -- 2013}
		{Graduação em Oceanografia}
		{Universidade Federal do Paraná}
		{\textbf{Monografia:} \textit{Variabilidade da Temperatura da Superfície do Mar durante um Evento Extremo de Precipitação em Santa Catarina}. Disponível para consulta \href{http://doi.org/10.13140/RG.2.2.15184.35847}{\textcolor{bleu_cite}{\textbf{aqui}}}. 
		\\ \textbf{Iniciação Científica: } \textit{Aspectos meteorológicos associados ao evento extremo de novembro de 2008 no leste do Estado de Santa Catarina}. Disponível para consulta \href{http://doi.org/10.13140/RG.2.2.25250.68800}{\textcolor{bleu_cite}{\textbf{aqui}}}.
		\\\texttt{GrADS}\slashsep\texttt{Python}\slashsep\texttt{Big data}\slashsep\texttt{Linux}}
\end{entrylist}

%----------------------------------------------------------------------------------------
%	ADDITIONAL INFORMATION
%----------------------------------------------------------------------------------------
\vspace{-0.4cm}

\begin{minipage}[t]{0.3\textwidth}
	\vspace{-\baselineskip} % Required for vertically aligning minipages
	
	\cvsect{Publicações importantes}
	
	\textbf{Livros e artigos} \\
	\textcolor{gray}{Python, modelagem numérica e treinamento de pessoal}
	 \begin{itemize}
		\itemsep-0.3em
		\item \textcolor{bleu_cite}{\href{http://mtc-m21c.sid.inpe.br/col/sid.inpe.br/mtc-m21c/2020/10.02.15.11/doc/publicacao.pdf}{COAWST User's Guide - 3° Ed.}}
		\item \textcolor{bleu_cite}{\href{http://mtc-m21c.sid.inpe.br/col/sid.inpe.br/mtc-m21c/2019/09.05.18.15/doc/publicacao.pdf}{Guia para utilização do COAWST - 2° Ed.}}
		\item \textcolor{bleu_cite}{\href{http://mtc-m21c.sid.inpe.br/col/sid.inpe.br/mtc-m21c/2018/09.13.17.45/doc/publicacao.pdf}{Guia para utilização do COAWST - 1° Ed.}}
		\item \textcolor{bleu_cite}{\href{https://link.springer.com/article/10.1007/s00703-020-00747-0}{Evento de chuva extrema Nordeste: um estudo de sensibilidade numérica}}
	 \end{itemize}  
\end{minipage}
\hfill
\begin{minipage}[t]{0.3\textwidth}
	\vspace{-\baselineskip} % Required for vertically aligning minipages

	\cvsect{Línguas}
	\vspace{-0.3cm}
	\begin{itemize}
	\itemsep0em
	\item \textbf{Português} - Nativo
	\item \textbf{Inglês} - Proficiente (\href{https://www.efset.org/cert/PrWqbW}{\textcolor{bleu_cite}{Nível C2}})
	\item \textbf{Italiano} - Intermediário 
	\item \textbf{Espanhol} - Iniciante
	\end{itemize}
	\vspace{0.8cm}
	
	\centering\includegraphics[width=0.3\textwidth]{qrcode.png}
\end{minipage}
\hfill
\begin{minipage}[t]{0.3\textwidth}
	\vspace{-\baselineskip} % Required for vertically aligning minipages

	\cvsect{Trabalho Voluntário}

	\textbf{\textcolor{bleu_cite}{\href{https://www.casaum.org/}{Casa 1}} } \hspace{1.cm} Jan 2019 --  Dez 2019\\
	\textcolor{gray}{Frente de Parcerias Comerciais e Empregabilidade}
	\begin{itemize}
		\item Suporte à eventos realizados pela organização
		\item Prospecção de possíveis parceiros potenciais para ajudar a financiar a organização
	\end{itemize}
\end{minipage}


%----------------------------------------------------------------------------------------

\end{document}
