%%%%%%%%%%%%%%%%%%%%%%%%%%%%%%%%%%%%%%%%%
% Developer CV
% LaTeX Template
% Version 1.0 (28/1/19)
%
% This template originates from:
% http://www.LaTeXTemplates.com
%
% Authors:
% Jan Vorisek (jan@vorisek.me)
% Based on a template by Jan Küster (info@jankuester.com)
% Modified for LaTeX Templates by Vel (vel@LaTeXTemplates.com)
%
% License:
% The MIT License (see included LICENSE file)
%
%%%%%%%%%%%%%%%%%%%%%%%%%%%%%%%%%%%%%%%%%

%----------------------------------------------------------------------------------------
%	PACKAGES AND OTHER DOCUMENT CONFIGURATIONS
%----------------------------------------------------------------------------------------

\documentclass[9pt]{developercv} % Default font size, values from 8-12pt are recommended
\definecolor{bleu_cite}{RGB}{0, 102, 204}
%----------------------------------------------------------------------------------------

\begin{document}

%----------------------------------------------------------------------------------------
%	TITLE AND CONTACT INFORMATION
%----------------------------------------------------------------------------------------

\begin{minipage}[t]{0.55\textwidth} % 45% of the page width for name
	\vspace{-\baselineskip} % Required for vertically aligning minipages
	
	% If your name is very short, use just one of the lines below
	% If your name is very long, reduce the font size or make the minipage wider and reduce the others proportionately
	\colorbox{black}{{\fontsize{24}{26}\textcolor{white}{\textbf{\MakeUppercase{Ueslei Adriano Sutil}}}}} % First name
	
	\vspace{6pt}
	
	{\Huge \textbf{Oceanógrafo}} % Career or current job title
\end{minipage}
\begin{minipage}[t]{0.235\textwidth} % 27.5% of the page width for the first row of icons
	\vspace{-\baselineskip} % Required for vertically aligning minipages
	
	% The first parameter is the FontAwesome icon name, the second is the box size and the third is the text
	% Other icons can be found by referring to fontawesome.pdf (supplied with the template) and using the word after \fa in the command for the icon you want
	\icon{MapMarker}{12}{São Paulo, SP}\\
	\icon{Phone}{12}{+55 12 99144-2815}\\

\end{minipage}
\begin{minipage}[t]{0.275\textwidth} % 27.5% of the page width for the second row of icons
	\vspace{-\baselineskip} % Required for vertically aligning minipages
	
	% The first parameter is the FontAwesome icon name, the second is the box size and the third is the text
	% Other icons can be found by referring to fontawesome.pdf (supplied with the template) and using the word after \fa in the command for the icon you want
	\icon{Linkedin}{12}{\textcolor{bleu_cite}{\href{https://www.linkedin.com/in/usutil/}{linkedin.com/in/usutil}}}\\
	\icon{At}{12}{\textcolor{bleu_cite}{\href{mailto:ueslei@outlook.com}{ueslei@outlook.com}}}
	%s\icon{Github}{12}{\textcolor{bleu_cite}{\href{https://github.com/uesleisutil}{github.com/uesleisutil}}}\\
	%\icon{Twitter}{12}{\textcolor{bleu_cite}{\href{https://twitter.com/uesleisutil}{@uesleisutil}}}\\
\end{minipage}

%----------------------------------------------------------------------------------------
%	INTRODUCTION, SKILLS AND TECHNOLOGIES
%-----------------------------------------------------Actively
\cvsect{\faUser\hspace{0.3cm} Sobre mim}

\begin{minipage}[t]{1\textwidth} % 40% of the page width for the introduction text
Sou oceanógrafo físico com habilidades em ciência, análise e visualização de dados. Possuo forte \textit{background} em investigação e extração de informações utilizando Python e implementa e operação de modelos de \textit{machine learning} na nuvem. 
\end{minipage}
\hspace{0.3cm}
\hfill % Whitespace between

\begin{center}
	\bubbles{4/Python,4/Jupyter,4/Jupyter,4/SQL,3.7/Growth, 3.35/LaTeX,3.2/AWS,3/Salesforce,3/HTML,3/CSS,2/Tableau,2/R}
\end{center}

%----------------------------------------------------------------------------------------
%	EXPERIENCE
%----------------------------------------------------------------------------------------
\vspace{-0.2cm}
\cvsect{\faBriefcase\hspace{0.3cm} Experiência profissional}

\begin{entrylist}
	\entry
		{2022 --  Hoje\\}
		{Especialista de CRM}
		{\href{https://www.esfera.com.vc}{\textcolor{bleu_cite}{\textbf{Esfera Fidelidade - Uma Empresa Santander}}}}
		{• Desenvolvimento de modelos utilizando \textit{machine learning} com foco no ciclo de vida do cliente para estudos e campanhas;\\
		• Estudos sobre o comportamento de clientes (Análise de \textit{cohort}, persona e tendências);\\ 
		• Segmentações de audiências para campanhas;\\
		• Ideação, condução e documentação de testes de \textit{Growth Hacking};\\
		• Produtização de dados de frentes de negócio com foco em tomada de decisões;\\
		\texttt{Python}\slashsep\texttt{PostgreSQL}\slashsep\texttt{Jupyter}\slashsep\texttt{AWS Athena e Sagemaker}\slashsep\texttt{Salesforce}\slashsep\texttt{Growth}}
		\\\entry
		{2021 --  2022\\}
		{Analista de CRM}
		{\href{https://www.esfera.com.vc}{\textcolor{bleu_cite}{\textbf{Esfera Fidelidade - Uma Empresa Santander}}}}
		{• Desenvolvimento de modelos estatísticos com foco na segmentação do cliente para disparos de campanhas;\\
		• Criação e sustentação de dashboards e indicadores operacionais;\\
		• Elaboração de estratégias de enriquecimento da base de clientes;\\
		• Segmentações de audiências para campanhas;\\
		• Qualificação e validação de e-mail marketing e notificações via aplicativo.\\
		\texttt{Python}\slashsep\texttt{Jupyter}\slashsep\texttt{PostgreSQL}\slashsep\texttt{Redash}\slashsep\texttt{Salesforce}}
		\\\entry
		{2017 -- 2021\\\footnotesize{Bolsa CNPq II}}
		{Pesquisador Assistente}
		{\href{https://www.gov.br/inpe/pt-br}{\textcolor{bleu_cite}{\textbf{Instituto Nacional de Pesquisas Espaciais (INPE)}}}}
		{• Construção da \textit{toolbox} model2roms. Projeto \textit{open-source} disponível no \href{https://github.com/uesleisutil/model2roms}{\textcolor{bleu_cite}{\textbf{Github}}}; \\	
		• Implementação dos modelos WaveWatch 3 e ROMS Sea-Ice em supercomputador;\\
		• Construção do \href{https://loa-inpe.github.io/}{\textcolor{bleu_cite}{\textbf{site}}} do Laboratório de Estudos do Oceano e da Atmosfera. \\
		\texttt{Python}\slashsep\texttt{LaTeX}\slashsep\texttt{Fortran}\slashsep\texttt{C++}\slashsep\texttt{Git}\slashsep\texttt{Bootstrap}\slashsep\texttt{HTML/CSS}}
		\\\entry
		{2016 -- 2017\\\footnotesize{Bolsa CNPq I}}
		{Pesquisador Assistente}
		{\href{https://www.gov.br/inpe/pt-br}{\textcolor{bleu_cite}{\textbf{Instituto Nacional de Pesquisas Espaciais (INPE)}}}}
		{•  Implementação do sistema de modelagem numérica COAWST em um supercomputador; \\
		 • Modelagem hidrodinâmica e atmosférica para estudar a interação oceano-atmosfera no Atlântico Sul. 
		\texttt{Python}\slashsep\texttt{Big data}\slashsep\texttt{Fortran}\slashsep\texttt{C++}\slashsep\texttt{MATLAB}}
\end{entrylist}

%----------------------------------------------------------------------------------------
%	EDUCATION
%----------------------------------------------------------------------------------------
\vspace{-0.5cm}
\cvsect{\faGraduationCap\hspace{0.3cm} Educação}

\begin{entrylist}
	\entry
		{2023 -- Hoje}
		{MBA em Data Science}
		{Faculdade de Informática e Administração Paulista (FIAP)}
		{}
		\entry
		{2014 -- 2016}
		{Mestrado em Sensoriamento Remoto}
		{Universidade Federal do Rio Grande do Sul (UFRGS)}
		{\href{https://lume.ufrgs.br/handle/10183/171223}{\textcolor{bleu_cite}{\textbf{Dissertação}}}: \textit{Interação Oceano-Atmosfera em um Ciclone Extratropical no Atlântico Sudoeste}.}
	\entry
		{2009 -- 2013}
		{Graduação em Oceanografia}
		{Universidade Federal do Paraná (UFPR)}
		{\href{http://doi.org/10.13140/RG.2.2.15184.35847}{\textcolor{bleu_cite}{\textbf{Monografia}}}: \textit{Variabilidade da Temperatura da Superfície do Mar durante um evento extremo de precipitação em Santa Catarina}. 
		\\ \href{http://doi.org/10.13140/RG.2.2.25250.68800}{\textcolor{bleu_cite}{\textbf{Iniciação científica}}}: \textit{Aspectos meteorológicos do evento extremo de novembro de 2008 em Santa Catarina}.
		}
\end{entrylist}

%----------------------------------------------------------------------------------------
%	ADDITIONAL INFORMATION
%----------------------------------------------------------------------------------------
\vspace{-0.4cm}

\begin{minipage}[t]{0.3\textwidth}
	\vspace{-\baselineskip} % Required for vertically aligning minipages
	
	\cvsect{\faBook\hspace{0.3cm} Publicações científicas}
	
	\textbf{Livro} %\\
	\textcolor{gray}{}
	 \begin{itemize}
		\item \small\textcolor{bleu_cite}{\href{http://mtc-m21c.sid.inpe.br/col/sid.inpe.br/mtc-m21c/2020/10.02.15.11/doc/publicacao.pdf}{COAWST User's Guide - 3° Ed.}}
	 \end{itemize}
	 \textbf{Artigo}
	 \begin{itemize}
		\item \small\textcolor{bleu_cite}{\href{https://www.nature.com/articles/s41598-021-89985-9}{Oceanic eddy-induced modifications to air-sea heat and CO2 fluxes in the Brazil-Malvinas Confluence}}
	 \end{itemize}  
\end{minipage}
\hfill
\begin{minipage}[t]{0.3\textwidth}
	\vspace{-\baselineskip} % Required for vertically aligning minipages

	\cvsect{\faCommentsO\hspace{0.3cm} Idiomas}
	\vspace{-0.3cm}
	\begin{itemize}
	\itemsep0em
	\item \textbf{Português} - Nativo
	\item \textbf{Inglês} - Proficiente (\href{https://www.efset.org/cert/PrWqbW}{\textcolor{bleu_cite}{Nível C2}})
	\item \textbf{Italiano} - Intermediário 
	\end{itemize}
	\vspace{1.3cm}
	
	\centering\tiny Atualizado em 26 de junho de 2023
\end{minipage}
\hfill
\begin{minipage}[t]{0.3\textwidth}
	\vspace{-\baselineskip} % Required for vertically aligning minipages

	\cvsect{\faChild\hspace{0.3cm} Trabalho Voluntário}

	\textbf{\textcolor{bleu_cite}{\href{https://www.casaum.org/}{Casa 1}} } \hspace{1.cm} Jan 2019 --  Dez 2019\\
	\textcolor{gray}{Frente de parcerias e empregabilidade}
	\begin{itemize}
		\item Suporte à eventos realizados pela organização;
		\item Prospecção de potenciais parceiros para a organização.
	\end{itemize}
\end{minipage}


%----------------------------------------------------------------------------------------

\end{document}
